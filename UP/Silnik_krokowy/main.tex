\documentclass{article}
\usepackage[utf8]{inputenc}
\usepackage{polski}
\usepackage{listings}
\usepackage{hyperref}
\usepackage{chngpage}
\usepackage{natbib}
\usepackage{graphicx}
\usepackage{url}
\usepackage{minted}

\setminted{
    linenos=true,
    autogobble,
    breaklines,
    frame=lines,
    framerule=1pt,
    framesep=10pt
}

\newenvironment{longlisting}{}{}

\makeatletter
\newcommand{\linia}{\rule{\linewidth}{0.4mm}}
\renewcommand{\maketitle}{\begin{titlepage}
    \vspace*{1cm}
    \begin{center}\small
    Politechnika Wrocławska\\
    Wydział Elektroniki\\
    Urządzenia Peryferyjne
    \end{center}
    \vspace{3cm}
    \noindent\linia
    \begin{center}
      \LARGE \textsc{\@title}
         \end{center}
     \linia
    \vspace{0.5cm}
    \begin{flushright}
    \begin{minipage}{7cm}
    \textit{\small Autor:}\\
    \normalsize \textsc{\@author} \par
    \end{minipage}
    \vspace{5cm}

     {\small środa, 14\textsuperscript{15}-17\textsuperscript{15} TN}\\
        Mgr inż. Kamil Szyc
     \end{flushright}
    \vspace*{\stretch{6}}
    \begin{center}
    \@date
    \end{center}
  \end{titlepage}
}
\makeatother
\author{225942 Justyna Skalska \\
        226126 Grzegorz Kopacz}
\title{\textbf{Urządzenia peryferyjne}\\
\normalsize{Sterowaniem silnikiem krokowym za pomocą USB}}

\begin{document}
\maketitle
\tableofcontents

\newpage

\section{Opis zadania}
Podczas zajęć laboratoryjnych mieliśmy zapoznać się ze sposobami sterowania silnikiem krokowym za pomocą USB. Pierwszym zadaniem było obrócenie wirnika o 360 stopni. Następnie trzeba było zmodyfikować napisany wcześniej program, tak aby wirnik wykonał jeden obrót w zdanym przez użytkownika czasie. Ostatnim zadaniem było dostosowanie prędkości obrotu wirnika do zadanej przez użytkownika oraz sprawienie, że wirnik nie startuje z wybraną prędkością lecz przyśpiesza do niej, aby uniknąć problemów z nagłym startem. 

\section{Wstęp} 
Silnik krokowy jest to silnik elektryczny, w którym impulsowe zasilanie prądem elektrycznym powoduje, że jego wirnik nie obraca się ruchem ciągłym, lecz wykonuje każdorazowo ruch obrotowy o ściśle określony kąt. Ze względu na typ budowy silniki krokowe można podzielić na: silnik z magnesem trwałym, silnik o zmiennej reluktancji oraz silnik hybrydowy. 
Podczas laboratorium nasze urządzenie było połączone z komputerem za pomocą portu USB. Za konwersje standardu USB na 8 bitową linię odpowiedzialny jest układ firmy FTDI o symbolu FT245BM. Do wyjścia tego układu, które ma postać 8 bitowej szyny dołączony jest układ ULN2803A.\cite{lab}

\section{Zadanie}
Kod został wykonany przy użyciu języka C\#. Silnik został podłączony do portu USB. Z przyczyn technicznych nie jestem w stanie zamieścić zdjęcia interfejsu wykonanej aplikacji. Posiadała ona jednak przyciski pozwalające wykonać obrót o zadany kąt oraz w zadanym czasie. Znajdował się tam także slider, dzięki któremu można było wybrać prędkość obrotu wirnika. Można go było wystartować oraz zatrzymać dzięki dodatkowym 2 przyciskom.
\newline
\newline

\begin{listing}
\caption{Zmienne globalne}
\begin{minted}{csharp}
// liczba wysłanych bitów
public UInt32 WrittenBytesCount = 0;
// bity pozwalające obrócić wirnik w prawo
public byte[] stepRightBytes = { 0x01, 0x08, 0x02, 0x04 };
// bity pozwalające obrócić wirnik w lewo
public byte[] stepLeftBytes = { 0x04, 0x02, 0x08, 0x01 };
// bity pozwalające rozłączyć się z urządzeniem
public byte[] disconnectBytes = { 0x00 };
// indeks wysłanych bitów
public int index = 0;
// procent prędkości maksymalnej wybranej przez użytkownika
public int speedPercentage = 0;
\end{minted}
\label{lst:global}
\end{listing}

\begin{listing}[H]
\caption{Funkcja do łączenia się z urządzeniem}
\begin{minted}{csharp}
public void connect() {
    UInt32 deviceCount = 0;
    Device = new FTDI();
    Device.GetNumberOfDevices(ref deviceCount);
    FTDI.FT_DEVICE_INFO_NODE[] deviceList = new FTDI.FT_DEVICE_INFO_NODE[deviceCount];
    Device.GetDeviceList(deviceList);
    Console.WriteLine("Device count: " + deviceList.Length);
    Device.OpenBySerialNumber(deviceList[0].SerialNumber);
    Device.SetBitMode(0xff, 1);
    foreach (FTDI.FT_DEVICE_INFO_NODE d in deviceList) {
        Console.WriteLine("Device name: " + d.Description);
        Console.WriteLine("Device serial number: " + d.SerialNumber);
    }
}
\end{minted}
\label{lst:connect}
\end{listing}

\begin{listing}[H]
\caption{Funkcja do odłączania urządzenia}
\begin{minted}{csharp}
public void disconnect() {
    WrittenBytesCount = 0;
    Int32 bytesToWrite = 1;
    Device.Write(disconnectBytes, bytesToWrite, ref WrittenBytesCount);
    Console.WriteLine("Device disconnected!");
}
\end{minted}
\label{lst:disconnect}
\end{listing}

\begin{listing}[H]
\caption{Funkcja wysyłania bitów do urządzenia}
\begin{minted}{csharp}
public void writeBytes(int speed, byte[] bytes) {
    byte[] x = { bytes[index % 4] };
    Device.Write(x, bytesToWrite, ref WrittenBytesCount);
    Thread.Sleep(speed);
    index++;
}
\end{minted}
\label{lst:write_bytes}
\end{listing}

\begin{listing}[H]
\caption{Funkcja wykonująca zadaną liczbę kroków wirnika}
\begin{minted}{csharp}
// obliczanie liczby kroków na obrót
var stepsCount = (int)(360 / 7.5f);
// obliczanie prędkości jednego obrotu
var timeout = (int)(5 * 1000 / stepsCount);
            
public void step(int count, int speed, byte[] bytes) {
    WrittenBytesCount = 0;
    for (int i = 0; i < count; i++) {
        writeBytes(speed, bytes);
    }
}
\end{minted}
\label{lst:step}
\end{listing}

\begin{listing}[H]
\caption{Funkcja wykonująca kroki wirnika przez określony czas}
\begin{minted}{csharp}
public void stepTimed(int time, int speed, byte[] bytes) {
    WrittenBytesCount = 0;

    Stopwatch timer = new Stopwatch();
    timer.Start();
    while (timer.Elapsed.TotalSeconds < time) {
        writeBytes(speed, bytes);
    }
    timer.Stop();
}
\end{minted}
\label{lst:step_timed}
\end{listing}

\begin{listing}[H]
\caption{Funkcja odpowiedzialna za obracanie wirnika z wybraną prędkością i pozwalająca na jego zatrzymanie}
\begin{minted}{csharp}
private void backgroundWorker1_DoWork(object sender, DoWorkEventArgs e) {
    BackgroundWorker worker = sender as BackgroundWorker;
    int minSpeed = 205;
    int maxSpeed = 5;
    int chosenSpeed = (int)(minSpeed - (((minSpeed - maxSpeed) * ch.speedPercentage) / 100));
    int speedStep = 10;
    int speed = minSpeed;

    while (true) {
        if (worker.CancellationPending == true) {
            e.Cancel = true;
            break;
        } else {
            ch.writeBytes(speed, ch.stepRightBytes);
            if (speed > maxSpeed) {
                speed -= speedStep;
            }
        }
    }

    ch.WrittenBytesCount = 0;
}
\end{minted}
\label{lst:background_worker}
\end{listing}

\section{Wnioski}
To ćwiczenie pozwoliło nam zapoznać się z podstawami obsługi silnika krokowego poprzez port USB. Okazało się to być stosunkowo prostym i przyjemnym zadaniem. Wiedza na ten temat może przydać się przy tworzeniu różnego rodzaju urządzeń, np. napędów DVD oraz drukarek, a także w szeroko rozumianej automatyce.
\newpage

\listoflistings
\bibliographystyle{unsrt}
\bibliography{references}
\end{document}
