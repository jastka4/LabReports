\documentclass{article}
\usepackage[utf8]{inputenc}
\usepackage{polski}
\usepackage{listings}
\usepackage{hyperref}
\usepackage{chngpage}
\usepackage{natbib}
\usepackage{graphicx}
\usepackage{url}
\usepackage{minted}

\setminted{
    linenos=true,
    autogobble,
    breaklines,
    frame=lines,
    framerule=1pt,
    framesep=10pt
}

\newenvironment{longlisting}{}{}

\makeatletter
\newcommand{\linia}{\rule{\linewidth}{0.4mm}}
\renewcommand{\maketitle}{\begin{titlepage}
    \vspace*{1cm}
    \begin{center}\small
    Politechnika Wrocławska\\
    Wydział Elektroniki\\
    Urządzenia Peryferyjne
    \end{center}
    \vspace{3cm}
    \noindent\linia
    \begin{center}
      \LARGE \textsc{\@title}
         \end{center}
     \linia
    \vspace{0.5cm}
    \begin{flushright}
    \begin{minipage}{7cm}
    \textit{\small Autor:}\\
    \normalsize \textsc{\@author} \par
    \end{minipage}
    \vspace{5cm}

    16 stycznia 2019\\
     {\small środa, 14\textsuperscript{15}-17\textsuperscript{15} TN}\\
        Mgr inż. Kamil Szyc
        
     \end{flushright}
    \vspace*{\stretch{6}}
    \begin{center}
    \end{center}
  \end{titlepage}
}
\makeatother
\author{225942 Justyna Skalska \\
        226126 Grzegorz Kopacz}
\title{\textbf{Urządzenia peryferyjne}\\
\normalsize{Rejestracja sygnałów (Advantech DAQ)}}

\begin{document}
\maketitle
\tableofcontents

\newpage

\section{Opis zadania}
Podczas zajęć laboratoryjnych mieliśmy podłączyć oraz przetestować działanie urządzenia Advantech USB-4702, generatora przebiegów oraz przetestować pobieranie danych. Kolejnym naszym zadaniem było napisanie programu w języku C\#, który miał działać jak oscyloskop. Kolejno nasz program miał zapisywać odebrane dane do pliku. Z odebranych przez nas danych mieliśmy następnie stworzyć wykres w wybranym przez nas programie. Do narysowania wykresów skorzystaliśmy z Matlaba oraz Microsoft Excel.

\section{Wstęp} 
Urządzenia DAQ jak na przykład używane podczas ćwiczenia Advantech USB-4702 to urządzenia pozwalające na przekonwertowanie sygnału analogowego na sygnał cyfrowy. Program Advantech Navigator daje możliwość odczytania odebranych sygnałów oraz analizy otrzymanych wykresów. 


\section{Zadanie}
Kod został wykonany przy użyciu języka C\#. Urządzenie Advantech USB-4702 zostało podłączone do portu USB. Niestety z przyczyn technicznych nie jesteśmy w stanie zamieścić zdjęcia interfejsu wykonanej przez nasz aplikacji. Posiadała ona jednak 2 przyciski: Start oraz Stop, które odpowiadały kolejno za rozpoczęcie pobierania danych z generatora przebiegów i zapisywania ich do pliku oraz oraz zaprzestanie wykonywania tych czynności.
\newline
\newline

\begin{listing}
\caption{Zmienne globalne}
\begin{minted}{csharp}
private string path = @"C:\Users\lab\Documents\MyTest.txt";
        static int numberDevice = 0, startChannel = 4, channels = 1, times = 0;
        static InstantAiCtrl ctrl = null;
        static double[] data = null;
        StreamWriter sw = null;
\end{minted}
\label{lst:global}
\end{listing}

\begin{listing}[H]
\caption{Kod do podłączenia się z urządzeniem}
\begin{minted}{csharp}
InitializeComponent();
            ctrl = new InstantAiCtrl();
            ctrl.SelectedDevice = new DeviceInformation(numberDevice);
            data = new double[channels];
            if (!ctrl.Initialized)
            {
                MessageBox.Show("Nie znaleziono urzadzenia");
                this.Close();
                return;
            }
\end{minted}
\label{lst:connect}
\end{listing}


\begin{listing}[H]
\caption{Funkcja zapisująca dane do pliku}
\begin{minted}{csharp}
private void timer1_Tick(object sender, EventArgs e)
        {
            ErrorCode error = ErrorCode.Success;
            error = ctrl.Read(startChannel, channels, data);
            if (error != ErrorCode.Success)
            {
                MessageBox.Show("Błąd");
                timer1.Stop();
                sw.Close();
                StartButton.Enabled = true;
                StopButton.Enabled = false;
            }
            else
            {
                sw.Write(" ");
                for (int j = 0; j < channels; j++)
                {
                    sw.Write(data[j]);
                }
                times++;
            }
        }
\end{minted}
\label{lst:write_bytes}
\end{listing}


\section{Wnioski}

Ćwiczenie pozwoliło nam przypomnieć sobie zasady działania generatora przebiegów czy różnic między sygnałami analogowymi a cyfrowymi. Dodatkowo mogliśmy bliżej poznać używane podczas ćwiczenia środowiska programistycznego DAQNavi.
\newpage

\listoflistings
\end{document}
