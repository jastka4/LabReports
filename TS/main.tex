\documentclass[12pt,a4paper,titlepage]{article}
\usepackage[utf8]{inputenc}
\usepackage{polski}
\usepackage[margin=1in]{geometry}
\usepackage{listings}
\usepackage{graphicx}
\usepackage{xcolor}
\usepackage{minted}
\usepackage{multirow}
\usepackage{ctable}
\usepackage{float}
\usepackage{tabu}
\usepackage{longtable}

\newcommand{\specialcell}[2][c]{%
  \begin{tabular}[#1]{@{}c@{}}#2\end{tabular}}
\newcommand{\specialcellbold}[2][c]{%
  \bfseries
  \begin{tabular}[#1]{@{}l@{}}#2\end{tabular}%
}
  
\makeatletter
\newcommand{\linia}{\rule{\linewidth}{0.4mm}}
\renewcommand{\maketitle}{\begin{titlepage}
    \vspace*{1cm}
    \begin{center}\small
    Politechnika Wrocławska\\
    Wydział Elektroniki\\
    Technologie Sieciowe
    \end{center}
    \vspace{3cm}
    \noindent\linia
    \begin{center}
      \LARGE \textsc{\@title}
         \end{center}
     \linia
    \vspace{0.5cm}
    \begin{flushright}
    \begin{minipage}{7cm}
    \textit{\small Autorzy:}\\
    \normalsize \textsc{\@author} \par
    \end{minipage}
    \vspace{5cm}

     {\small wtorek, 17\textsuperscript{05}-18\textsuperscript{45} TN}\\
        Dr inż. Michał Kucharzak
     \end{flushright}
    \vspace*{\stretch{6}}
    \begin{center}
    \@date
    \end{center}
  \end{titlepage}%
}
\makeatother
\author{Justyna Skalska, 225942 \\
Bartosz Powęska, 234720
}
\title{Projekt sieci}

\begin{document}
\maketitle
\tableofcontents
\newpage

\section{Wstęp}
Niniejsza dokumentacja jest projektem lokalnej sieci komputerowej dla firmy „G.I. Industries” zajmującej się produkcją oprogramowania znajdującego zastosowanie w obsłudze specjalistycznych robotów używanych w procesie produkcji pojazdów militarnych. Projekt został opracowany na podstawie dokumentacji dostarczonej przez zleceniodawcę i jest przystosowany na przewidywany wzrost liczby pracowników. W pracach nad projektem kierowano się przede wszystkim niezawodnością (przy przewidywanych obciążeniach) i łatwością rozbudowy sieci w przyszłości.
\section{Inwentaryzacja zasobów}

Siedziba firmy składa się z dwóch znajdujących się blisko siebie budynków. Pierwszy budynek jest dwupiętrowy, drugi natomiast jest parterowcem. Pomiędzy nimi nie znajdują się żadne zabudowania. W firmie są trzy grupy robocze: zarząd i kadry, programiści i testerzy oraz administratorzy.

\begin{table}[H]
    \centering
    \begin{tabular}{c|c|c|c|c|}
    \cline{2-5}
    & \multicolumn{4}{c|}{\specialcellbold{Liczba użytkowników}} \\ \cline{2-5}
    & \multicolumn{3}{c|}{\specialcellbold{Budynek A}} & \multicolumn{1}{c|}{\specialcellbold{Budynek B}} \\ \hline 
         \multicolumn{1}{|c|}{\specialcellbold{Grupa robocza}} & \specialcellbold{Parter} & \specialcellbold{Piętro 1} & \specialcellbold{Piętro 2} & \specialcellbold{Parter} \\ \hline
         \multicolumn{1}{|c|}{Zarząd i kadry} & - & 6 & 16 & 12 \\ \hline
         \multicolumn{1}{|c|}{Programiści i testerzy} & - & 49 & 31 & 68 \\ \hline
         \multicolumn{1}{|c|}{Administratorzy} & 4 & - & - & 2 \\ \hline
         & \multicolumn{4}{c|}{\specialcellbold{Liczba drukarek}} \\ \cline{2-5}
         & 1 & 2 & 2 & 2 \\ \cline{2-5}
         & \multicolumn{4}{c|}{\specialcellbold{Kamery IP}} \\ \cline{2-5}
         & 8 & 5 & 5 & 8 \\ \cline{2-5}
         & \multicolumn{4}{c|}{\specialcellbold{Laboratorium}} \\ \cline{2-5}
         & 16 & - & - & - \\ \cline{2-5}
    \end{tabular}
    \caption{Podstawowa inwentaryzacja zasobów przedsiębiorstwa}
    \label{tab:inwentaryzacja_zasobow}
\end{table}

\begin{table}[H]
    \centering
    \begin{tabular}{|c|c|} \hline
         \specialcellbold{Rodzaj zasobu} & \specialcellbold{Liczba} \\ \hline
         Komputer & 188 \\ \hline
         Drukarka & 7 \\ \hline
         Kamery IP & 26 \\ \hline
         Roboty & 16 \\ \hline 
    \end{tabular}
    \caption{Obecne zasoby firmy}
    \label{tab:zasoby_firmy}
\end{table}

\begin{table}[H]
    \centering
    \begin{tabular}{|c|c|} \hline
        \specialcellbold{Rodzaj programu} & \specialcellbold{Przykład} \\ \hline
        Przeglądarka internetowa & Mozilla Firefox, Google Chrome \\ \hline
        Komunikator & Discord, HipChat \\ \hline
        Klient FTP & FileZilla, Cyberduck \\ \hline
        VoIP & Discord, HipChat  \\ \hline
        Wideokonferencja & Zoom, Google Hangouts \\ \hline
    \end{tabular}
    \caption{Aplikacje wykorzystywane w firmie}
    \label{tab:aplikacje_w_firmie}
\end{table}

\begin{table}[H]
    \centering
    \begin{tabular}{|c|c|c|} \hline
        \specialcellbold{Oznaczenie} & \specialcellbold{Lokalizacja} & \specialcellbold{Podłączone punkty abonencie} \\ \hline
        MDF  & Budynek A, Parter   & Budynek A, Parter \\ \hline
        IDF1 & Budynek A, Piętro 1 & Budynek A, Piętro 1 \\ \hline
        IDF2 & Budynek A, Piętro 2 & Budynek A, Piętro 2 \\ \hline
        IDF3 & Budynek B, Parter   & Budynek B \\ \hline
    \end{tabular}
    \caption{Punkty dystrybucyjne}
    \label{tab:punkty_dystrybucyjne}
\end{table}

\section{Analiza potrzeb użytkowników}

\begin{table}
    \centering
    \begin{tabular}{c|c|c|c|c|} \cline{2-5}
        & \multicolumn{4}{c|}{\specialcellbold{Transfer (down/up), kb/s}} \\ \hline
        \multicolumn{1}{|c|}{\specialcellbold{Użytkownik/\\Aplikacja}} & \specialcellbold{Przeglądarka} & \specialcellbold{Praca w \\ chmurze} & \specialcellbold{Komunikator} & \specialcellbold{Wideorozmowy} \\ \hline
        \multicolumn{1}{|c|}{Zarząd i kadry} & 80/15 & 23/36 & 15/15 & 40/40 \\ \hline
        \multicolumn{1}{|c|}{\specialcell{Programiści \\i testerzy}} & 110/10 & 30/53 & 15/15 & 40/40 \\ \hline
        \multicolumn{1}{|c|}{Administratorzy} & 100/20 & 20/30 & 15/15 & 10/10 \\ \hline
        \multicolumn{1}{|c|}{Sieć gości} & 20/10 & 5/5 & 5/5 & - \\ \hline
    \end{tabular}
    \caption{Przepływ danych jednego użytkownika z/do Internetu}
    \label{tab:transfer_osoba_internet}
\end{table}

\begin{table}[H]
    \centering
    \begin{tabular}{|c|c|c|c|c|c|c|} \hline
        \multirow{2}{*}{\specialcellbold{Punkt\\dystrybucyjny}} & \multicolumn{2}{c|}{\specialcellbold{Grupa robocza}} & \multicolumn{4}{c|}{\specialcellbold{Internet (kb/s)}} \\ \cline{2-7}
        & \specialcellbold{Nazwa} & \specialcellbold{Ilość\\stanowisk} & DOWN & UP & \specialcell{UP\\łącznie} & \specialcell{DOWN\\łącznie} \\ \hline
        \multirow{5}{*}{MDF}
        & Zarząd i kadry                        & - & 158 & 106 &   - &   - \\ \cline{2-7}
        & \specialcell{Programiści\\i testerzy} & - & 195 & 118 &   - &   - \\ \cline{2-7}
        & Administratorzy                       & 4 & 145 &  75 & 580 & 300 \\ \cline{2-7}
        & Sieć gości                            & - &  30 &  20 &   - &   - \\ \cline{2-7}
        & Łącznie                               & 4 & - & - & 580 & 300 \\ \hline
        
        \multirow{5}{*}{IDF1}
        & Zarząd i kadry                        & 16 & 158 & 106 & 2528 & 1696 \\ \cline{2-7}
        & \specialcell{Programiści\\i testerzy} & 31 & 195 & 118 & 6045 & 3658 \\ \cline{2-7}
        & Administratorzy                       & - & 145 & 75 & - & - \\ \cline{2-7}
        & Sieć gości                            & - & 30 & 20 & - & - \\ \cline{2-7}
        & Łącznie                               & 4 & - & - & 8573 & 5354 \\ \hline
        
        \multirow{5}{*}{IDF2}
        & Zarząd i kadry                        & 6 & 158 & 106 & 948 & 636 \\ \cline{2-7}
        & \specialcell{Programiści\\i testerzy} & 49 & 195 & 118 & 9555 & 5782 \\ \cline{2-7}
        & Administratorzy                       & - & 145 & 75 & - & - \\ \cline{2-7}
        & Sieć gości                            & - & 30 & 20 & - & - \\ \cline{2-7}
        & Łącznie                               & 4 & - & - & 10503 & 6418 \\ \hline
        
        \multirow{5}{*}{IDF3}
        & Zarząd i kadry                        & 12 & 158 & 106 & 1896 & 1272 \\ \cline{2-7}
        & \specialcell{Programiści\\i testerzy} & 68 & 195 & 118 & 13260 & 8024 \\ \cline{2-7}
        & Administratorzy                       & 2 & 145 & 75 & 290 & 150 \\ \cline{2-7}
        & Sieć gości                            & - & 30 & 20 & - & - \\ \cline{2-7}
        & Łącznie                               & 4 & - & - & 15446 & 9446 \\ \hline
    \end{tabular}
    \caption{Transfer danych z/do Internetu}
    \label{tab:transfer_internet}
\end{table}

\begin{table}[H]
    \centering
    \begin{tabular}{|c|c|c|c|c|c|c|} \hline
        \multirow{2}{*}{\specialcellbold{Punkt\\dystrybucyjny}} & \multicolumn{2}{c|}{\specialcellbold{Grupa robocza}} & \multicolumn{4}{c|}{\specialcellbold{Serwer plików 1 (kb/s)}} \\ \cline{2-7}
        & \specialcellbold{Nazwa} & \specialcellbold{Ilość\\stanowisk} & DOWN & UP & \specialcell{UP\\łącznie} & \specialcell{DOWN\\łącznie} \\ \hline
        \multirow{2}{*}{MDF}
        & Administratorzy   &  4 & 8000 &  600 & 32000 &  2400 \\ \cline{2-7}
        & Kamery            &  8 &  110 & 2800 &   880 & 22400 \\ \cline{2-7}
        & Łącznie           & 12 & - & - & 32880 & 24800 \\ \hline
        
        \multirow{2}{*}{IDF1}
        & Administratorzy   & - & 8000 &  600 &   - &     - \\ \cline{2-7}
        & Kamery            & 5 &  110 & 2800 & 550 & 14000 \\ \cline{2-7}
        & Łącznie           & 5 & - & - & 550 & 14000 \\ \hline
        
        \multirow{2}{*}{IDF2}
        & Administratorzy   & - & 8000 &  600 &   - &     - \\ \cline{2-7}
        & Kamery            & 5 &  110 & 2800 & 550 & 14000 \\ \cline{2-7}
        & Łącznie           & 5 & - & - & 550 & 14000 \\ \hline
        
        \multirow{2}{*}{IDF3}
        & Administratorzy   & 2 & 8000 &  600 & 16000 &  1200 \\ \cline{2-7}
        & Kamery            & 8 &  110 & 2800 &   880 & 22400 \\ \cline{2-7}
        & Łącznie           & 5 & - & - & 16880 & 23400 \\ \hline
        
    \end{tabular}
    \caption{Transfer danych z/do serwera plików 1}
    \label{tab:transfer_serwer_1}
\end{table}

\begin{table}[H]
    \centering
    \begin{tabular}{|c|c|c|c|c|c|c|} \hline
        \multirow{2}{*}{\specialcellbold{Punkt\\dystrybucyjny}} & \multicolumn{2}{c|}{\specialcellbold{Grupa robocza}} & \multicolumn{4}{c|}{\specialcellbold{Serwer plików 2 (kb/s)}} \\ \cline{2-7}
        & \specialcellbold{Nazwa} & \specialcellbold{Ilość\\stanowisk} & DOWN & UP & \specialcell{UP\\łącznie} & \specialcell{DOWN\\łącznie} \\ \hline
        
        \multirow{4}{*}{MDF}
        & Zarząd i kadry                            & - & 600 & 550 &    - &    - \\ \cline{2-7}
        & \specialcell{Programiści\\i testerzy}     & - & 700 & 590 &    - &    - \\ \cline{2-7}
        & Administratorzy                           & 4 & 800 & 300 & 3200 & 1200 \\ \cline{2-7}
        & Łącznie                                   & 4 & 800 & 300 & 3200 & 1200 \\ \hline
        
        \multirow{4}{*}{IDF1}
        & Zarząd i kadry                            &  6 & 600 & 550 &  3600 &  3300 \\ \cline{2-7}
        & \specialcell{Programiści\\i testerzy}     & 49 & 700 & 590 & 34300 & 28910 \\ \cline{2-7}
        & Administratorzy                           &  - & 800 & 300 &     - &     - \\ \cline{2-7}
        & Łącznie                                   & 55 &   - &   - & 37900 & 32210 \\ \hline
        
        \multirow{4}{*}{IDF2}
        & Zarząd i kadry                            & 16 & 600 & 550 &  9600 &  8800 \\ \cline{2-7}
        & \specialcell{Programiści\\i testerzy}     & 31 & 700 & 590 & 21700 & 18290 \\ \cline{2-7}
        & Administratorzy                           &  - & 800 & 300 &     -  &     - \\ \cline{2-7}
        & Łącznie                                   & 47 &   - &   - & 31300 & 32210 \\ \hline
        
        \multirow{4}{*}{IDF3}
        & Zarząd i kadry                            & 12 & 600 & 550 &  7200 &  6600 \\ \cline{2-7}
        & \specialcell{Programiści\\i testerzy}     & 68 & 700 & 590 & 47600 & 40120 \\ \cline{2-7}
        & Administratorzy                           & 2  & 800 & 300 &  1600 &   600 \\ \cline{2-7}
        & Łącznie                                   & 82 & -   & -   & 31300 & 32210 \\ \hline
        
    \end{tabular}
    \caption{Transfer danych z/do serwera plików 2}
    \label{tab:transfer_serwer_2}
\end{table}

\begin{table}[H]
    \centering
    \begin{tabular}{|c|c|c|c|c|c|c|} \hline
        \multirow{2}{*}{\specialcellbold{Punkt\\dystrybucyjny}} & \multicolumn{2}{c|}{\specialcellbold{Grupa robocza}} & \multicolumn{4}{c|}{\specialcellbold{Serwer WWW (kb/s)}} \\ \cline{2-7}
        & \specialcellbold{Nazwa} & \specialcellbold{Ilość\\stanowisk} & DOWN & UP & \specialcell{UP\\łącznie} & \specialcell{DOWN\\łącznie} \\ \hline
        
        \multirow{4}{*}{MDF}
        & Zarząd i kadry                            & - & 330 & 440 &   - &   - \\ \cline{2-7}
        & \specialcell{Programiści\\i testerzy}     & - & 380 & 430 &   - &   - \\ \cline{2-7}
        & Administratorzy                           & 4 & 380 & 390 & 1520 & 1560 \\ \cline{2-7}
        & Łącznie                                   & 4 &   - &   - & 1520 & 1560 \\ \hline
        
        \multirow{4}{*}{IDF1}
        & Zarząd i kadry                            &  6 & 230 &  45 &  1380 &  270 \\ \cline{2-7}
        & \specialcell{Programiści\\i testerzy}     & 49 & 190 &  35 &  9310 & 1715 \\ \cline{2-7}
        & Administratorzy                           &  - & 210 &  60 &     - &    - \\ \cline{2-7}
        & Łącznie                                   & 55 &   - &   - & 10690 & 1985 \\ \hline
        
        \multirow{4}{*}{IDF2}
        & Zarząd i kadry                            & 16 & 230 &  45 &  5280 &  7040 \\ \cline{2-7}
        & \specialcell{Programiści\\i testerzy}     & 31 & 190 &  35 & 11780 & 13330 \\ \cline{2-7}
        & Administratorzy                           &  - & 210 &  60 &     - &     - \\ \cline{2-7}
        & Łącznie                                   & 47 &   - &   - & 17060 & 20370 \\ \hline
        
        \multirow{4}{*}{IDF3}
        & Zarząd i kadry                            & 12 & 230 &  45 &  3960 &  5280 \\ \cline{2-7}
        & \specialcell{Programiści\\i testerzy}     & 68 & 190 &  35 & 25840 & 29240 \\ \cline{2-7}
        & Administratorzy                           &  2 & 210 &  60 &   760 &   780 \\ \cline{2-7}
        & Łącznie                                   & 82 &   - &   - & 30560 & 35300 \\ \hline
        
    \end{tabular}
    \caption{Transfer danych z/do serwera WWW}
    \label{tab:transfer_serwer_www}
\end{table}

\begin{table}[H]
    \centering
    \begin{tabular}{|c|c|c|c|c|c|c|} \hline
        \multirow{2}{*}{\specialcellbold{Punkt\\dystrybucyjny}} & \multicolumn{2}{c|}{\specialcellbold{Grupa robocza}} & \multicolumn{4}{c|}{\specialcellbold{Serwer pocztowy (kb/s)}} \\ \cline{2-7}
        & \specialcellbold{Nazwa} & \specialcellbold{Ilość\\stanowisk} & DOWN & UP & \specialcell{UP\\łącznie} & \specialcell{DOWN\\łącznie} \\ \hline
        
        \multirow{4}{*}{MDF}
        & Zarząd i kadry                            & - & 230 &  45 &   - &   - \\ \cline{2-7}
        & \specialcell{Programiści\\i testerzy}     & - & 190 &  35 &   - &   - \\ \cline{2-7}
        & Administratorzy                           & 4 & 210 &  60 & 840 & 240 \\ \cline{2-7}
        & Łącznie                                   & 4 &   - &   - & 840 & 240 \\ \hline
        
        \multirow{4}{*}{IDF1}
        & Zarząd i kadry                            &  6 & 230 &  45 &  1380 &  270 \\ \cline{2-7}
        & \specialcell{Programiści\\i testerzy}     & 49 & 190 &  35 &  9310 & 1715 \\ \cline{2-7}
        & Administratorzy                           &  - & 210 &  60 &     - &    - \\ \cline{2-7}
        & Łącznie                                   & 55 &   - &   - & 10690 & 1985 \\ \hline
        
        \multirow{4}{*}{IDF2}
        & Zarząd i kadry                            & 16 & 230 &  45 & 3680 &  720 \\ \cline{2-7}
        & \specialcell{Programiści\\i testerzy}     & 31 & 190 &  35 & 5890 & 1085 \\ \cline{2-7}
        & Administratorzy                           &  - & 210 &  60 &    - &    - \\ \cline{2-7}
        & Łącznie                                   & 47 &   - &   - & 9570 & 1805 \\ \hline
        
        \multirow{4}{*}{IDF3}
        & Zarząd i kadry                            & 12 & 230 &  45 &  2760 &  540 \\ \cline{2-7}
        & \specialcell{Programiści\\i testerzy}     & 68 & 190 &  35 & 12920 & 2380 \\ \cline{2-7}
        & Administratorzy                           &  2 & 210 &  60 &   420 &  120 \\ \cline{2-7}
        & Łącznie                                   & 82 &   - &   - & 16100 & 3040 \\ \hline
        
    \end{tabular}
    \caption{Transfer danych z/do serwera pocztowy}
    \label{tab:transfer_serwer_pocztowy}
\end{table}

\begin{table}[H]
    \centering
    \begin{tabular}{|c|c|c|c|c|c|c|} \hline
        \multirow{2}{*}{\specialcellbold{Punkt\\dystrybucyjny}} & \multicolumn{2}{c|}{\specialcellbold{Grupa robocza}} & \multicolumn{4}{c|}{\specialcellbold{Drukarka}} \\ \cline{2-7}
        & \specialcellbold{Nazwa} & \specialcellbold{Ilość\\stanowisk} & DOWN & UP & \specialcell{UP\\łącznie} & \specialcell{DOWN\\łącznie} \\ \hline
        
        \multirow{4}{*}{MDF}
        & Zarząd i kadry                            & - & 10 &  180 &  - &   - \\ \cline{2-7}
        & \specialcell{Programiści\\i testerzy}     & - & 10 &  170 &  - &   - \\ \cline{2-7}
        & Administratorzy                           & 4 & 10 &  175 & 40 & 700 \\ \cline{2-7}
        & Łącznie                                   & 4 &  - &    - & 40 & 700 \\ \hline
        
        \multirow{4}{*}{IDF1}
        & Zarząd i kadry                            &  6 & 10 &  180 &  60 & 1080 \\ \cline{2-7}
        & \specialcell{Programiści\\i testerzy}     & 49 & 10 &  170 & 490 & 8330 \\ \cline{2-7}
        & Administratorzy                           &  - & 10 &  175 &   - &    - \\ \cline{2-7}
        & Łącznie                                   & 55 &   - &   - & 550 & 9410 \\ \hline
        
        \multirow{4}{*}{IDF2}
        & Zarząd i kadry                            & 16 & 10 &  180 & 160 & 2880 \\ \cline{2-7}
        & \specialcell{Programiści\\i testerzy}     & 31 & 10 &  170 & 310 & 5270 \\ \cline{2-7}
        & Administratorzy                           &  - & 10 &  175 &   - &    - \\ \cline{2-7}
        & Łącznie                                   & 47 &   - &   - & 470 & 8150 \\ \hline
        
        \multirow{4}{*}{IDF3}
        & Zarząd i kadry                            & 12 & 10 & 180 & 120 &  2160 \\ \cline{2-7}
        & \specialcell{Programiści\\i testerzy}     & 68 & 10 & 170 & 680 & 11560 \\ \cline{2-7}
        & Administratorzy                           &  2 & 10 & 175 &  20 &   350 \\ \cline{2-7}
        & Łącznie                                   & 82 &  - &   - & 820 & 14070 \\ \hline
        
    \end{tabular}
    \caption{Transfer danych z/do drukarki}
    \label{tab:transfer_drukarka}
\end{table}

\begin{table}[H]
    \centering
    \begin{tabular}{c|c|c|c|} \cline{2-3}
        & \multicolumn{2}{c|}{\specialcellbold{Internet (kb/s)}} \\ \cline{2-4}
        & \specialcellbold{Down} & \specialcellbold{Up} & \specialcellbold{Sesje} \\ \hline
        \multicolumn{1}{|c|}{Serwer WWW}      &  80 & 170 & 96 \\ \hline
        \multicolumn{1}{|c|}{Serwer pocztowy} & 890 & 410 & 12 \\ \hline
    \end{tabular}
    \caption{Transfer danych pomiędzy serwerami i Internetem}
    \label{tab:transfer_internet_serwery}
\end{table}

\begin{table}[H]
    \centering
    \begin{tabular}{|c|c|} \hline
        \multicolumn{2}{|c|}{\specialcellbold{Transfer danych (kb/s)}} \\ \hline
        \specialcellbold{Down} & \specialcellbold{Up} \\ \hline
        27200 & 27200 \\ \hline
    \end{tabular}
    \caption{Transfer danych pomiędzy sieciami laboratoryjną i biurową}
    \label{tab:transfer_lab_biuro}
\end{table}

\section{Założenia projektowe}
Założenia projektowe obejmują wdrożenie sieci w siedzibie firmy "G.I. Industries", która mieści się w dwóch budynkach połączonych kablem światłowodowym. Sieć będzie nowoczesna i łatwa w rozbudowie. W budynkach należących do przedsiębiorstwa zostało zainstalowane okablowanie strukturalne (kat.  6) wraz z niezbędnymi szafami teleinformatycznymi oraz wszystkie  urządzenia  końcowe (serwery,  drukarki, komputery,  kamery  IP,  itp.), które należy podłączyć do sieci.
W budynkach znajdują się cztery punkty dystrybucyjne z szafami teleinformatycznymi, do których schodzą się kable z obsługiwanych przez nie gniazd z różnych pięter i budynków. Odległość pomiędzy punktem dystrybucyjnym MDF a punktami IDF w każdym budynku nie przekracza 90 metrów,  pomiędzy  MDF  a  wszystkimi  IDF  w  obrębie  każdego  budynku  zainstalowano  okablowanie miedziane (skrętka) kat. 6.
\begin{itemize}
    \item MDF - główny punkt dystrybucyjny zlokalizowany na parterze budynku A obsługuje ruch generowany przez 4 użytkowników,
    \item IDF1 - pomocniczy punkt dystrybucyjny zlokalizowany na pierwszym piętrze budynku A obsługuje ruch generowany przez 55 użytkowników,
    \item IDF2 - pomocniczy punkt dystrybucyjny znajdujący się na drugim piętrze budynku A obsługujący 47 użytkowników
    \item IDF3 - pomocniczy punkt dystrybucyjny zlokalizowany na parterze budynku B obsługujący 82 użytkowników
\end{itemize}
Użytkownicy podzieleni są na 3 grupy robocze: administratorzy, programiści i testerzy oraz zarząd i kadry.\bigskip\\
Do punktów dystrybucyjnych należy także sprzęt należący do klienta:
\begin{itemize}
    \item serwer WWW,
    \item serwer pocztowy,
    \item serwer plików 1,
    \item serwer plików 2,
    \item 2 punkty dostępowe WiFi,
    \item 7 drukarek,
    \item 26 kamer IP,
    \item 16 robotów.
\end{itemize}
Konieczna jest instalacja i konfiguracja serwerów, przełączników, routera oraz punktów dostępowych.

Główne założenia projektowe:
\begin{itemize}
    \item Okablowanie poziome oraz szkieletowe znajdujące się wewnątrz budynków będzie wykonane w standardzie Gigabit Ethernet 1000Base-T (skrętka kat. 6),
    \item Połączenie między budynkami będzie wykorzystywało kable światłowodowe jednomodowe,
    \item Sieć będzie posiadać co najmniej 20\% wolnych portów na przełącznikach, aby umożliwić przyszłą rozbudowę spowodowaną wzrostem liczby pracowników,
    \item Technologia VLAN pozwoli na ograniczenie ilości burz broadcastowych, zwiększenie bezpieczeństwa sieci oraz umożliwi logiczne grupowanie stacji końcowych, które są fizycznie rozproszone w sieci
    \item Zapewnienie odpowiedniej konfiguracji sieci bezprzewodowej i kontroli dostępu oraz odseparowana sieci dla gości. Do szyfrowania sieci WiFi zostanie użyty standard szyfrowania WPA2-Enterprise,
    \item Niezawodność połączenia z internetem zostanie zapewniona dzięki dzierżawie łączy od dwóch niezależnych od siebie operatorów
    \item Serwery obsługujące ruch internetowy zostaną odizolowane w strefie DMZ, aby  w przypadku  włamania,  atakujący  nie  uzyskał  dostępu  do  całej  sieci komputerowej przedsiębiorstwa,
    \item Ochrona  przed  dostępem do wewnętrznych zasobów z zewnątrz zostanie osiągnięta poprzez prywatną adresację,
    \item Odporność na fizyczne uszkodzenia będzie zapewniony dzięki redundancji zasobów
\end{itemize}

\newpage
\listoftables
\end{document}