\documentclass[12pt,a4paper,titlepage]{article}
\usepackage[utf8]{inputenc}
\usepackage{polski}
\usepackage{listings}
\usepackage{graphicx}
\usepackage{xcolor}
\usepackage{minted}
\usepackage{amsmath}
\usepackage{caption}
\usepackage{hyperref} 
 
\setminted{
    autogobble,
    breaklines,
    framerule=1pt,
    framesep=10pt
}

\newenvironment{longlisting}{}{}

\makeatletter
\newcommand{\linia}{\rule{\linewidth}{0.4mm}}
\renewcommand{\maketitle}{\begin{titlepage}
    \vspace*{1cm}
    \begin{center}\small
    Politechnika Wrocławska\\
    Wydział Elektroniki\\
    Bezpieczeństwo Systemów i Usług Informatycznych 2
    \end{center}
    \vspace{3cm}
    \noindent\linia
    \begin{center}
      \LARGE \textsc{\@title}
         \end{center}
     \linia
    \vspace{0.5cm}
    \begin{flushright}
    \begin{minipage}{7cm}
    \textit{\small Autor:}\\
    \normalsize \textsc{\@author} \par
    \end{minipage}
    \vspace{5cm}

     {\small wtorek, 11\textsuperscript{15}-14\textsuperscript{00} TN}\\
        mgr inż. Przemysław Świercz
     \end{flushright}
    \vspace*{\stretch{6}}
    \begin{center}
    \@date
    \end{center}
  \end{titlepage}%
}
\makeatother
\author{Justyna Skalska, 225942}
\title{Sprawozdanie nr 2\\
(Hackme oraz Hackme 2.0)}

\begin{document}

\maketitle

\section{Omówienie tematu}
Naszym zadaniem podczas laboratorium było przejście dwóch gier, w których wcielamy się w hackera i próbujemy zalogować się do systemy łamiąc hasła na różne ciekawe sposoby. Adresy stron:
\begin{itemize}
    \item \url{http://uw-team.org/hackme/}
    \item \url{http://uw-team.org/hm2/}
\end{itemize}

\section{Omówienie rozwiązania}

\subsection{Hackme}
\subsubsection{Poziom 1}
Hasło było jawnie zapisane w tagu \mintinline{HTML}{<script>} znajdującym się po tagu \mintinline{HTML}{<body>}.
\subsubsection{Poziom 2}
Hasło w zostało zapisane w zmiennej, w pliku "script.js" i wykorzystane w tagu \mintinline{HTML}{<script>}, w kodzie strony do sprawdzenia poprawności.
\subsubsection{Poziom 3}
W tagu \mintinline{HTML}{<script>} znajdował się kod do sprawdzania poprawności hasła, który łączył wybrane ciągi znaków. Wystarczyło przejrzeć te ciągi oraz ich wybrane podciągi.
\subsubsection{Poziom 4}
W tagu \mintinline{HTML}{<script>} znajdował się kod wykonujący zadane równanie. Hasło było jego wynikiem. Wystarczyło obliczyć równanie zapisane w kodzie i wpisać je jako hasło.
\subsubsection{Poziom 5}
Trzeba było wstawić takie wartości, aby równanie przedstawione poniżej było prawdziwe. Wybrałam liczbę sekund równą 42 i liczbę pomocniczą równą 1.
\begin{listing}[H]
\begin{minted}{JS}
861 = ((seconds * (seconds - 1)) / 2) * (document.getElementById('pomoc').value % 2)
\end{minted}
\end{listing}
\subsubsection{Poziom 6}
Zadanie to polegało na znajomości funkcji \mintinline{js}{substring}. Trzeba było zauważyć, że zmienna i w pętli \mintinline{js}{for} powiększa się o dwa co daje tylko 3 obroty pętli zamiast pięciu. Dzięki temu łatwo można było obliczyć każdy poszczególny podciąg zadany w pętli. Następnie trzeba było odnaleźć zadany podciąg z utworzonego wcześniej ciągu znaków i połączyć go z nim. Dzięki temu można było w łatwy sposób otrzymać hasło. Wymagało to tylko odrobiny skupienia i prześledzenia kodu.
\subsubsection{Poziom 7}
Wystarczyło znaleźć jak zakodowane jest wymienione w kodzie strony hasło. Każda litera hasła odpowiadała jednej literze ciągu, który mieliśmy wpisać w pole tekstowe.
\subsubsection{Poziom 8}
Część zmiennych znajdowała się w innym pliku, podanym w skrypcie. Trzeba było obliczyć wartość tych zmiennych i użyć ich podczas tworzenia hasła. W pętli for głównego skryptu trzeba było obliczyć zmienną \mintinline{js}{i} oraz zmienną pomocniczą \mintinline{js}{qet} i na ich podstawie znaleźć numer znaku z zadanego ciągu, który chcieliśmy użyć w haśle.

\subsection{Hackme 2.0}
\subsubsection{Poziom 1}
Znajdował się tam ukryty \mintinline{HTML}{<input>}, którego wartość wynosiła "text". Aby hasło było poprawne wartość w obu inputach musiała być taka sama. Wystarczyło więc wpisać tekst z ukrytego formularza do tego nieukrytego.
\subsubsection{Poziom 2}
Wystarczyło dekodować podany w skrypcie ciąg znaków i wpisać go jako hasło.
\subsubsection{Poziom 3}
Hasło można było znaleźć zamieniając zapisany ciąg zer i jedynek na liczbę dziesiętną.
\subsubsection{Poziom 4}
Podczas tego zadania w tle działał skrypt, którego nie dało się zatrzymać inaczej niż wpisując hasło lub "X" co powodowało zakończenie całego Hackme. W inspektorze nie można było dostać kodu strony. Aby go znaleźć weszłam w zakładkę "Sieci" i skopiowałam odpowiedź żądania o kod strony z przeglądarki. W odpowiedzi był skrypt, dzięki któremu łatwo było już obliczyć hasło.
\subsubsection{Poziom 5}
Trzeba było zmienić nazwy inputów w formularzu z "login" na "log" i "haslo" na "has", a następnie w obu podać liczbę 1. Alternatywnie można było podmienić dane w linku, ponieważ był to \mintinline{HTML}{GET} i wszystkie dane były wysyłane jawnie.
\subsubsection{Poziom 6}
Należało znaleźć ciasteczko i przeczytać jaki adres posiadała kolejna strona w zagadce, a następnie do niej przejść.
\subsubsection{Poziom 7}
W tym zadaniu trzeba było pobrać kod strony z żądania \mintinline{HTML}{GET}, które zostało wysłane podczas ładowania strony. Znajdował się w nim skrypt, dzięki któremu dostałam się do podfolderu na serwerze o nazwie "includes". W nim znajdował się plik "cosik.js", którego nazwa była hasłem tego etapu.
\subsubsection{Poziom 8}
Używając podglądu zapytań można było dostać cały kod strony. Aby dostać się do niej należało ustawić w nagłówku żądania referera jako "\url{http://www.onet.pl/}". Dodatkowo w kodzie pobranym wcześniej można było odczytać jawnie zapisane hasło. Używając referera oraz odczytanego hasła stworzyłam zapytanie, które w odpowiedzi zawierało nazwę strony z kolejnym etapem.
\subsubsection{Poziom 9}
W ostatnim zadaniu treść strony pojawiała się tylko o godzinie 1:00. Ja skorzystałam z przesyłanego przez przeglądarkę zapytania \mintinline{HTML}{GET} i dostałam się w ten sposób do kodu strony. Alternatywnie można było przestawić zegarek systemowy tak, aby wskazywał godzinę 1:00. W treści znajdowało się kilka zakodowanych ciągów znaków. Pierwszy z nich był zakodowany binarnie, wystarczyło skorzystać z wyszukiwarki Google by znaleźć stronę konwertującą ciąg binarny na ASCII. Kolejne dwa ciągi były zakodowane przy użyciu szyfru przesuwającego 13 oraz 2.

\section{Wnioski}
Podczas zajęć udało mi się wykonać wszystkie zadania z obu stron. Był to bardzo ciekawy trening pokazujący w jak łatwy sposób można dostać się do kodu strony, zrozumieć go, a także nim manipulować. Właśnie z takich powodów niektórzy twórcy decydują się na zaciemnianie (ang. \textit{obfuscation}) kodu swoich witryn internetowych, które znacznie utrudnia jego czytanie i zrozumienie. W trakcie wykonywania tego ćwiczenia przydały mi się wcześniej nabyte umiejętności związane z językiem JavaScript, a także znajomość RESTa oraz narzędzi przeglądarki. Te zadania można było także wykonać używając cURL, Postmana, SoapUI lub innego podobnego narzędzia.

\end{document}