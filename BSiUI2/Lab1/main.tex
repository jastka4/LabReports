\documentclass[12pt,a4paper,titlepage]{article}
\usepackage[utf8]{inputenc}
\usepackage{polski}
\usepackage{listings}
\usepackage{graphicx}
\usepackage{xcolor}
\usepackage{minted}
\usepackage{amsmath}
\usepackage{caption}
 
\setminted{
    linenos=true,
    autogobble,
    breaklines,
    frame=lines,
    framerule=1pt,
    framesep=10pt
}

\newenvironment{longlisting}{}{}

\makeatletter
\newcommand{\linia}{\rule{\linewidth}{0.4mm}}
\renewcommand{\maketitle}{\begin{titlepage}
    \vspace*{1cm}
    \begin{center}\small
    Politechnika Wrocławska\\
    Wydział Elektroniki\\
    Bezpieczeństwo Systemów i Usług Informatycznych 2
    \end{center}
    \vspace{3cm}
    \noindent\linia
    \begin{center}
      \LARGE \textsc{\@title}
         \end{center}
     \linia
    \vspace{0.5cm}
    \begin{flushright}
    \begin{minipage}{7cm}
    \textit{\small Autor:}\\
    \normalsize \textsc{\@author} \par
    \end{minipage}
    \vspace{5cm}

     {\small wtorek, 11\textsuperscript{15}-14\textsuperscript{00} TN}\\
        mgr inż. Przemysław Świercz
     \end{flushright}
    \vspace*{\stretch{6}}
    \begin{center}
    \@date
    \end{center}
  \end{titlepage}%
}
\makeatother
\author{Justyna Skalska, 225942}
\title{Sprawozdanie nr 1\\
(Komunikator z szyfrowaniem)}

\begin{document}

\maketitle
\section{Omówienie tematu}
Naszym zadaniem podczas laboratorium było stworzenie prostego komunikatora (chatu) o architekturze klient-serwer wspierającego bezpieczną wymianę sekretu (protokół Diffie-Hellman) oraz obsługującego zadany format komunikacji.
Komunikaty wymieniane były w formacie JSON i podzielone były na przedstawione poniżej etapy.

\begin{table}[h!]
\caption{Format danych oparty na JSONie}
\begin{tabular}{|c|c|c|}
\hline
Etap & A (klient)                           & B (serwer) \\
\hline
1     & \{ “request”: “keys” \} →            & \\
\hline
2     &                                      & ← \{ “p”: 123, “g”: 123 \} \\
\hline
3     & \{ “a”: 123 \} →                     & ← \{ “b”: 123 \} \\
\hline
4     & \{ “encryption”: “none” \} →         & \\
\hline
5     & \{ “msg”: “...”, “from”: “John” \} → & ← \{ “msg”: “...”, “from”: “Anna” \} \\
\hline
\end{tabular}
\end{table}

Komunikat zawierający szyfrowanie jest opcjonalny. Jeśli klient nie wyśle tego argumentu jego domyślą wartością jest brak szyfrowania. Może on zawierać jedną z trzech typów:
\begin{itemize}
    \item Brak szyfrowania ("none"),
    \item Szyfr Cezara ("caesar"),
    \item Szyfrowanie XOR jednobajtowe - kluczem jest najmłodszy bajt sekretu ("xor").
\end{itemize}{}

Każda treść wiadomości zapisana jest w formacie base64, aby uniknąć problemów z niektórymi znakami przesyłanymi między klientem a serwerem.

\section{Omówienie rozwiązania oraz kodu}
Zaimplementowane przeze mnie rozwiązanie korzysta z biblioteki "org.json" do parsowania komunikatów wysyłanych w formacie JSON. Aplikacja może szyfrować wiadomości w trzech różnych trybach szyfrowania. Funkcje szyfrujące i deszyfrujące informacje przedstawione są poniżej.

\begin{listing}[H]
\caption{Funkcje do szyfrowania i deszyfrowania wiadomości}
\begin{minted}{Java}
protected String decrypt(final String message) {
    if (Encryption.CAESAR.equals(encryption)) {
        final int shift = key.intValue() % 26;
        return CaesarCipher.encryptDecrypt(message, 26 - shift);
    } else if (Encryption.XOR.equals(encryption)) {
        final byte[] bytes = key.toByteArray();
        return XOREncryption.encryptDecrypt(message, (char) bytes[bytes.length - 1]);
    } else {
        return message;
    }
}

protected String encrypt(final String message) {
    if (Encryption.CAESAR.equals(encryption)) {
        final int shift = key.intValue() % 26;
        return CaesarCipher.encryptDecrypt(message, shift);
    } else if (Encryption.XOR.equals(encryption)) {
        final byte[] bytes = key.toByteArray();
        return XOREncryption.encryptDecrypt(message, (char) bytes[bytes.length - 1]);
    } else {
        return message;
    }
}
\end{minted}
\end{listing}

Funkcja przedstawiona powyżej korzysta ze stworzonych klas implementujących szyfrowanie XOR lub Cezara. Oba szyfrowania przedstawione są poniżej.

\begin{listing}[H]
\caption{Klasa szyfrująca wiadomości szyfrowaniem XOR}
\begin{minted}{Java}
public class XOREncryption {
    private XOREncryption() {}

    public static String encryptDecrypt(final String message, final char key) {
        final StringBuilder outputString = new StringBuilder();
        final int len = message.length();

        for (int i = 0; i < len; i++) {
            outputString.append((char) (message.charAt(i) ^ key));
        }

        return outputString.toString();
    }
}
\end{minted}
\end{listing}

\begin{listing}[H]
\caption{Klasa szyfrująca wiadomości szyfrem Cezara}
\begin{minted}{Java}
public class CaesarCipher {
    private CaesarCipher() {}

    public static String encryptDecrypt(final String text, final int shift) {
        final StringBuilder result = new StringBuilder();

        for (int i = 0; i < text.length(); i++) {
            if (Character.isUpperCase(text.charAt(i))) {
                char ch = (char) (((int) text.charAt(i) +
                        shift - 65) % 26 + 65);
                result.append(ch);
            } else {
                char ch = (char) (((int) text.charAt(i) +
                        shift - 97) % 26 + 97);
                result.append(ch);
            }
        }
        return result.toString();
    }
}
\end{minted}
\end{listing}

Dla każdego klienta i serwera należy wygenerować klucz prywatny, które będzie wykorzystywany do obliczania klucza końcowego. Każdy klient oraz serwer mają swoje niepowtarzane, generowane podczas połączenia klucze prywatne. Generowanie kluczy P oraz Q po stronie serwera może zająć nawet do 30 sekund, a wartości P oraz Q są ze sobą ściśle powiązane.

\begin{listing}[H]
\caption{Funkcja do generowania klucza prywatnego}
\begin{minted}{Java}
protected BigInteger generatePrivateKey(final SecureRandom secureRandom, final BigInteger p) {
    boolean good = false;
    BigInteger b = null;
    while (!good) {
        b = new BigInteger(1025, secureRandom);
        if (b.compareTo(p.subtract(BigInteger.ONE)) < 0) {
            good = true
        };
    }
    return b;
}
\end{minted}
\end{listing}

\begin{listing}[H]
\caption{Fragment kodu generujący parametry P oraz Q serwera}
\begin{minted}{Java}
private BigInteger findPrimitive(final BigInteger p, final BigInteger q) {
    BigInteger g = BigInteger.ONE;
    boolean foundPrimitive = false;

    while (!foundPrimitive) {
        g = g.add(BigInteger.ONE);
        BigInteger remainder = g.remainder(p);
        if (BigInteger.ONE.compareTo(remainder) != 0) {
            BigInteger gToQModP = g.modPow(q, p);
            if (BigInteger.ONE.compareTo(gToQModP) != 0) {
                foundPrimitive = true;
            }
        }
    }
    return g;
}
\end{minted}
\end{listing}

\begin{listing}[H]
\caption{Fragment kodu generujący parametry P oraz Q serwera}
\begin{minted}{Java}
final SecureRandom secureRandom = SecureRandom.getInstance("SHA1PRNG");
BigInteger p = null;
BigInteger q = null;

boolean probablePrime = false;
while (!probablePrime) {
    q = BigInteger.probablePrime(1024, secureRandom);
    p = q.shiftLeft(1);
    p = p.setBit(0);
    probablePrime = p.isProbablePrime(100);
}

final BigInteger g = findPrimitive(p, q);
final BigInteger secretKey = generatePrivateKey(secureRandom, p);
\end{minted}
\end{listing}

\section{Rezultat prac i wnioski}
Udało mi się zaimplementować wszystkie wymienione w omówieniu funkcjonalności. Jedyną która sprawiła mi lekkie problemy było odbieranie opcjonalnego komunikatu zawierającego sposób szyfrowania.
Wymiana kluczy w protokole Diffiego-Hellmana jest bardzo prosta w implementacji, a wygenerowany klucz jest praktycznie niemożliwy do odtworzenia przez osobę podsłuchującą.

\end{document}
