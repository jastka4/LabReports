\begin{tikzpicture}
        \umlstateinitial[name=begin]
        \node[singlestate] at (4,0) (getId){Pobierz identyfikator produktu};
        \node[singlestate] at (4,-2) (getUser){Pobierz identyfikator klienta};
        \node[singlestate] at (4,-4) (checkAmount){Sprawdź, czy dany przedmiot jest w magazynie};
        \umlstatedecision[x=4,y=-7,name=check]
        \node[singlestate] at (0,-7) (success){Do koszyka klienta dodaj przedmiot};
        \node[singlestate] at (8,-7) (error){Wyświetl komunikat o braku towaru};
        \node[singlestate] at (0,-9) (removeFromStock){Zmniejsz ilość produktu w magazynie};
        \node[singlestate] at (0,-11) (countValue){Przelicz wartość koszyka};
        \umlstateexit[x=0,y=-12.5,name=exit1]
        \umlstateexit[x=8,y=-8.5,name=exit2]
        
        \umltrans{begin}{getId}
        \umltrans{getId}{getUser}
        \umltrans{getUser}{checkAmount}
        \umltrans{checkAmount}{check}
        \umltrans[arg=Jest, pos=0.5]{check}{success}
        \umltrans[arg=Brak, pos=0.3]{check}{error}
        \umltrans{success}{removeFromStock}
        \umltrans{removeFromStock}{countValue}
        \umltrans{countValue}{exit1}
        \umltrans{error}{exit2}
        
    \end{tikzpicture}